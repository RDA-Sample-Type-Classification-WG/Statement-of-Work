\documentclass{scrartcl}
\usepackage[english]{babel}
\usepackage[utf8]{inputenc}
\usepackage[T1]{fontenc}
\usepackage{graphicx}
\usepackage[colorlinks=true,urlcolor=blue]{hyperref}
\usepackage{tweaktitle}

\begin{document}

\titlehead{\hfill\includegraphics[width=8em]{rda-logo}}
\title{Sample Type Classification WG}
\subtitle{Statement of Work}

\maketitle

\begin{abstract}\noindent
  This document formulates the Statement of Work for the RDA Sample
  Type Classification Working Group in the course of formation,
  conforming to the RDA guidelines for creating a working group.
\end{abstract}

\section{WG Charter}

The working group will develop a recommendation for one or more
vocabularies for global, cross-domain classification of material
sample types.  The vocabularies are intended to cover all types of
samples being investigated in the sciences, for all scientific
disciplines and domains.  This includes natural and human-made
physical objects, but also digital representations of those being used
in computer simulations.  The scope of the vocabularies will be
limited to the top-level sample type classification.  The detailed
description of the samples will be left to existing domain-specific
vocabularies.

The working group intends to follow the guidelines presented in Cox et
al, 2021\footnote{Cox SJD, Gonzalez-Beltran AN, Magagna B, Marinescu
  M-C (2021) Ten simple rules for making a vocabulary FAIR. PLoS
  Comput Biol 17(6): e1009041.
  \url{https://doi.org/10.1371/journal.pcbi.1009041}} for implementing
FAIR vocabularies.


\section{Value Proposition}

In sciences, raw data is often collected from experimental or
computational data acquisitions made on physical or simulated objects.
In many cases, information about the experimental or computational
techniques and the parameters of the data acquisition can be captured
automatically by the device or the control software.  But information
about the subject under investigation, i.e.\ the sample, needs to be
acquired by other means, typically manually, and, as a result, in many
cases is not well represented in the dataset’s metadata.  Without
information documenting sample acquisition and processing prior to
analysis, the data produced is neither interoperable nor reusable.
That is not FAIR!  We need better sample descriptions in our datasets.

Many kinds of samples are the subject of analysis. Some examples:
\begin{itemize}
\item single crystals, crystal powder, novel functional materials,
\item complex structures: solar cells, batteries,
\item macromolecular crystals, e.g.\ proteins,
\item archaeological artefacts, fossils, minerals, rocks,
\item artwork, e.g.\ paintings,
\item living plants, environmental samples,
\item virtual samples generated for computational modeling (e.g.\
  alloy microstructures).
\end{itemize}

As a result, any schema for the sample description at global level can
only be very generic.

Many metadata standards for sample descriptions are available.
However, most of them are limited to a particular type of sample
within a specific scientific domain and are not applicable to others.
Given the huge variety of physical objects investigated in the
sciences, it is not realistic to come up with a common overarching
standard that would be applicable to describe all kinds of samples.

This creates a challenge for repositories that need to host sample
descriptions from all domains.  They would need to classify the sample
first and know its type in order to know which standard is suitable to
describe it.  Such a sample classification would need a standardized
vocabulary that should apply to any physical sample being
investigated.  An overarching vocabulary that is generally applicable
needs to be interoperable, and to connect to domain specific
vocabularies.

The top-level classification of sample types should be shallow but
very broad, applicable to all sorts of objects from all domains.  It
is not intended to replace domain specific vocabularies, but to
connect them.

\section{Engagement with existing work in the area}

Prior to formulating this Statement of Work, the group did some
research in order to find existing vocabularies that could be used to
classify samples.  We only found some that are specific to a
particular scientific domain respectively, but none that would be
domain agnostic, i.e.\ generally applicable to all physical objects
being investigated in any scientific domain.

The working group will operate under the auspices of the RDA Physical
Samples and Collections in the Research Data Ecosystem Interest
Group.\footnote{\url{https://www.rd-alliance.org/groups/physical-samples-and-collections-research-data-ecosystem-ig/}}
The group will start by collecting existing vocabularies in use to
categorize samples, for example:
\begin{itemize}
\item iSamples (\url{https://isamplesorg.github.io/vocabularies/})
\item Biodiversity Information Standards (TDWG) sample type
  (\url{https://www.tdwg.org/community/osr/material-sample/})
\item DiSSCo (\url{https://www.dissco.eu/})
\item the GeoSciML vocabularies of the Commission for Geoscience
  Information of the International Union of Geological Sciences
  (\url{http://geosciml.org/resource/def/voc/})
\item Battery Interface Ontology (BattINFO,
  \url{https://www.big-map.eu/dissemination/battinfo})
\item Geoscience Australia (\url{https://vocabs.ga.gov.au/vocab/})
\end{itemize}

The ExPaNDS project has developed the Photon and Neutron Experimental
Techniques (PaNET) ontology.\footnote{Collins, S. P., da Graça Ramos,
  S., Iyayi, D., Görzig, H., González Beltrán, A., Ashton, A., Egli,
  S., \& Minotti, C. (2021). ExPaNDS ontologies
  v1.0. Zenodo. \url{https://doi.org/10.5281/zenodo.4806026}}
This ontology classifies experimental techniques being used in large
scale Photon and Neutron facilities according to four fundamental
properties: the experimental physical process, the experimental probe,
the functional dependency and the purpose.  The working group will
consider learning from PaNET as a model on how to generate a logical
structure in a complex realm of concepts based on fundamental
properties.

The working group will also consider the InteroperAble Descriptions of
Observable Property Terminology (I-ADOPT,
\url{https://i-adopt.github.io/ontology/index.html}) and learn from
their approach to construction of the ontology framework, development
of use cases and terminology catalogue spanning across scientific
domains, as well as both human and machine readable implementation.

\section{UN Sustainable Development Goals (SDGs)}

The vocabularies for sample type classification developed by the
working group will contribute to make the data collected from
experimental measurements on physical samples FAIR.  While there is no
direct relation between this work to any of the SDGs, it is assumed
that the science that is enabled by FAIR data will have an impact on
implementing the SDGs in general.

\section{Work Plan}

The work that the group needs to do can be roughly described by the
following task list:
\begin{itemize}
\item Create a GitHub repository for sharing activities, creating
  issues, and maintaining versions.
\item Collect use cases, providing an overview of things that we need
  to look at.
\item Identify existing candidate cross-domain vocabularies
  (e.g.\ iSamples).
\item Harmonize the candidates to generate a proposal for draft
  vocabularies.  Implement (e.g.\ in SKOS) and publish the draft(s) in
  the GitHub repository.
\item Test the draft vocabularies by mapping sample descriptions from
  existing sample registration databases.  Working group participants
  will identify existing repositories of sample descriptions.  Working
  group participants will work on mapping for sample descriptions in
  their domain of expertise.
\item Generate GitHub issues to report gaps, inconsistencies, or
  suggested modifications.
\item Working group meets virtually on a regular basis to discuss and
  resolve issues and make decisions on updates to the draft
  vocabulary.
\item Iterate test, review, update cycle until WG agrees that a
  workable recommendation has been developed.
\item Draft recommendation document.
\item Publish the product on a public vocabulary service with
  resolvable URIs.
\end{itemize}

Assuming the formal work period of the group to start with IDW 2025,
this work plan can be split into the following milestones:
\begin{description}
\item[M0, RDA P25 / IDW 2025] Kick off for the group, present the
  group and its goals, discuss the work plan, involve additional
  contributors.
\item[M6, RDA P26] present use cases and ``landscape analysis'',
  e.g.\ an overview of existing vocabularies that we need to take into
  account and build upon, discuss plans for vocabularies.
\item[M12, RDA P27] present draft vocabularies, gather feedback from
  the community, incite early adopters to test the draft.
\item[M18, RDA P28] wrap up, present the draft recommendation document
  and first adoption cases, transition to maintenance mode.
\end{description}

\section{Adoption Plan}

The working group will engage with sample repositories and sample
metadata aggregators (e.g.\ iSamples, IGSN/DataCite, DiSSCo) as part
of the vocabulary development process, with the intention that the WG
include participants responsible for interoperable sample descriptions
in their purview.  Such involvement should motivate application of the
product vocabularies by these systems.

Frictionless access to vocabularies in whole or as individual terms
via a public, web-based vocabulary service will minimize the effort
required to adopt and use vocabularies.

WG operation and publication via GitHub will foster participation in
maintenance and update of the vocabularies, whether by the RDA
Physical Samples and Collections in the Research Data Ecosystems IG,
Earth Science Information Partners (ESIP), or some other persistent
organization.

Covering all of science is clearly a mammoth task, even in terms of a
high-level overview.  The working group intends to make the vocabulary
as complete as possible, but recognises that future expansion will be
necessary.  Such expansion will also encourage adoption, as any
under-representation of domains can be rectified so no sample is left
out.

\section{Initial Membership}

\begin{itemize}
\item David Elbert (co-chair), Johns Hopkins University
\item Heike Görzig,
  Helmholtz-Zentrum Berlin für Materialien und Energie
\item Katherine Rial,
  Helmholtz-Zentrum Berlin für Materialien und Energie
\item Kirsten Elger (co-chair),
  GFZ Helmholtz Centre for Geosciences, Vice President of IGSN e.V.
\item Kirsty Syder, Diamond Light Source
\item Lesley Wyborn,
  National Computational Infrastructure, member of ESIP Sample Curation group
\item Natalie Raia,
  University of Arizona, member of ESIP Sample Curation group
\item Petra ten Hoopen (co-chair), British Antarctic Survey
\item Rolf Krahl (co-chair),
  Helmholtz-Zentrum Berlin für Materialien und Energie
\item Rossella Aversa, Karlsruhe Institute of Technology
\item Stephen Richard,
  U.S. Geoscience Information Network, iSamples,
  member of ESIP Sample Curation group
\item Steve Collins, Diamond Light Source
\end{itemize}

\end{document}
